% !TeX spellcheck = si_SI
\chapter{Rezultati}\label{cha:rezultati}

V tem poglavju predstavite \textbf{ugotovljena dejstva}, torej rezultate vaših meritev, analiz, preračunov ipd. Če je naloga obsežnejša in sestavljena iz več ločenih sklopov, lahko rezultate iz posameznega sklopa predstavljate tudi v ločenih (pod)poglavjih. Končna oblika mora biti takšna, da je karseda pregledna, jasna in razumljiva.
